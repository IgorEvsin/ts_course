\section{Что такое (хороший) прогноз}
\label{sec:forecast_quality}

\defn{Определение (из Википедии)}{
  Прогноз (от греч. «предвидение, предсказание») — это
  научно обоснованное суждение о возможных состояниях объекта в
  будущем и (или) об альтернативных путях и сроках их
  осуществления. В узком смысле, это вероятностное суждение о
  будущем состоянии объекта исследования.
}

Другими словами, прогноз - это изучение наблюдаемых
(реально существующих!) закономерностей и их экстраполяция в будущее.
Чудес не бывает. Чем точнее вы выделите и опишете эти закономерности,
тем точнее будет прогноз. Чем лучше вы оцените погрешность
экстраполяции каждой составляющей, тем более адекватной получится
оценка погрешности прогноза в целом.

Очевидно, некоторые вещи прогнозируются лучше, чем другие. Время восхода солнца
мы научились предсказывать довольно точно, а вот колебания цен акций
все еще остаются для нас загадкой (возможно, навсегда).

На сложность составления прогноза влияют следующие факторы:

\begin{enumerate}
  \item Как хорошо мы понимаем факторы, влияющие на объект исследования
  \item Какое количество данных доступно
  \item Насколько будущее похоже на прошлое
  \item Влияют ли наши прогнозы на поведение объекта, которое мы
    пытаемся спрогнозировать (частный случай - самоисполняющееся пророчество)
\end{enumerate}

TODO: What can be forecast? Взять 4 пункта и написать, что делать, если они
не выполняются. Для judgmental forecasting дать ссылку на
суперфоркастинг и, возможно, на темы, которые там не покрываются
(если такие есть). Футурология.

Статистическую модель же (в самом общем понимании этого слова) можно
считать хорошей при соблюдении двух условий:

\begin{enumerate}
  \item Во-первых, в остатках не должно быть явных закономерностей. Попросту
    говоря, остатки должны быть случайны. Это значит, что наша модель уже
    учла все, что можно (нужно?). В модели не должно быть смещения по
    какой-то из фичей, а смещение по времени должно иметь объяснимый характер.
  \item Хороший прогноз - это на самом деле НЕ точный прогноз
    (как многие ошибочно думают), а прогноз с достоверно известной
    погрешностью, причем очень желательно - минимально возможной для
    данного ряда.
\end{enumerate}

Последнее условие очень важно. На протяжении всего курса мы будем вас
предостерегать не полагаться исключительно на статистические критерии
(в случае временных рядов они будут очень часто лгать). Дело в том,
что хорошее качество на тестовой выборке можно получить очень
нечестными путями, например через переобучение на тестовой выборке.

Best worst corner case

Именно поэтому мы убеждены, что количество параметров модели должно
быть минимально возможным (но не меньше!). По крайней мере, оно
должно быть кратно меньше, чем число степеней
свободы данных. Иначе возникает риск переподгонки, что ничуть не
менее опасно, чем неправильная модель. Правильная оценка числа
степеней свободы особенно важна, если в данных есть внутренние
взаимозависимостии. Ведь если они сильны, то реальное число степеней
свободы может быть много меньше, чем формальное количество значений
данных.

\ex{}{Кстати, известный баг с ложными корреляциями - это типичный
  пример именно такой ситуации (значения временного ряда
    взаимозависимы, поэтому фактическое число степеней свободы на порядки
меньше, чем количество точек данных).}

В качестве награды за большое количество прочитанного материала
наградим вас вашей первой моделью временного ряда

\defn{Наивный прогноз}{

  Для наивного прогноза мы берем предыдущее значение ряда в
  качества прогноза для следующего шага.

  \[\hat{y}_{T+h|T} = y_T.\]
}

\ex{Наивный прогноз}{

  Не смейтесь над простотой этого метода.

  Во-первых, он служит хорошим бейзлайном для ориентирования в
  метриках вашей модели временного ряда. Делать выводы на основании
  MSE или любой другой выбранной метрики (подробнее в следующей
  секции) на бэктесте довольно сложно
  без какого-то референсного значения.

  Во-вторых, его можно посчитать для любого ряда и очень быстро.

  Во-третьих, когда вы будете работать с реальными данными, вы
  удивитесь, насколько в некоторых случаях будет незначительна
  разница между прогнозами сложных моделей и прогнозом подобного
  наивного метода.

  Наивный метод хорошо работает для временных рядов, порожденных
  \href{https://en.wikipedia.org/wiki/Martingale_(probability_theory)}{мартингалами}.
}

Можно ли придумать модель с меньшим числом свобод, чем наивный
прогноз? Вот то-то.