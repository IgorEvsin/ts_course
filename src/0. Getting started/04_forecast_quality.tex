\section{Что такое (хороший) прогноз}

\url{https://habr.com/ru/articles/821231/comments/#comment_26942703}

\defn{Определение из википедии}{
  Прогноз (от греч. «предвидение, предсказание») — это
  научно обоснованное суждение о возможных состояниях объекта в
  будущем и (или) об альтернативных путях и сроках их
  осуществления. В узком смысле, это вероятностное суждение о
  будущем состоянии объекта исследования.
}

Некоторые вещи прогнозируются лучше, чем другие. Время восхода солнца
мы научились предсказывать довольно точно, а вот колебания цен акций
все еще остаются для нас загадкой. На сложность составления прогноза
влияют следующие факторы:

\begin{enumerate}
  \item Как хорошо мы понимаем факторы, влияющие на объект исследования
  \item Какое количество данных доступно
  \item Насколько будущее похоже на прошлое
  \item Влияют ли наши прогнозы на поведение объекта, которое мы
    пытаемся спрогнозировать (частный случай - самоисполняющееся пророчество)
\end{enumerate}

What can be forecast? Взять 4 пункта и написать, что делать, если они
не выполняются. Для judgmental forecasting дать ссылку на
суперфоркастинг и, возможно, на темы, которые там не покрываются
(если такие есть). Футурология.

Описать, с чем мы будем работать — случаи, когда данные всё-таки есть.

Надо уделить большее внимание анализу ошибок моделей.

Ведь что такое прогноз временного ряда? Это изучение наблюдаемых
(реально существующих!) закономерностей и их экстраполяция в будущее.
Чудес не бывает. Чем точнее вы выделите и опишете эти закономерности,
тем точнее будет прогноз. Чем лучше вы оцените погрешность
экстраполяции каждой составляющей, тем более адекватной получится
оценка погрешности прогноза в целом.

Не полагаться только на статистические критерии! Тем более, что они
почти всегда будут лгать.

Статистическую модель (в самом общем понимании этого слова) можно
считать хорошей при соблюдении двух условий:

\begin{enumerate}
  \item Во-первых, в остатках не должно быть явных закономерностей. Попросту
    говоря, остатки должны быть случайны. Это значит, что наша модель уже
    учла все, что можно (нужно?).
  \item Ведь хороший прогноз - это на самом деле НЕ точный прогноз
    (как многие ошибочно думают), а прогноз с достоверно известной
    погрешностью, причем очень желательно - минимально возможной для
    данного ряда.
\end{enumerate}

Смещения - зависимость ошибок от фичей или от времени?
Как измерять погрешность.best worst corner case

И, во-вторых, количество параметров модели должно быть минимальным.
По крайней мере, оно должно быть кратно меньше, чем число степеней
свободы данных. Иначе возникает риск сверхподгонки, что ничуть не
менее опасно, чем неправильная модель! Правильная оценка числа
степеней свободы особенно важна, если в данных есть внутренние
взаимозависимостии. Ведь если они сильны, то реальное число степеней
свободы может быть много меньше, чем формальное количество значений
данных. Кстати, известный баг с ложными корреляциями - это типичный
пример именно такой ситуации (значения временного ряда
  взаимозависимы, поэтому фактическое число степеней свободы на порядки
меньше, чем количество точек данных).

Мартингал. Наивный прогноз.
Задание: построить наивный прогноз для простого ряда. Можно
ненавязчиво включать разные официальные источники данных в курс.