\section{Что вообще можно делать с рядами}

Вот вы сидите на собеседование и эксцентричный интервьюер вам
говорит: "Представь, что у тебя есть временной ряд, что бы ты с ним
мог сделать". Это реальный вопрос, который задали на реальном
собеседовании одному из авторов данного курса. Поэтому если вы
попадете в подобную ситуацию, то не забудьте:

В случае одномерных рядов:

\begin{enumerate}
  \item Прогнозирование - попытка сделать точное и стабильное
    предположение о будущем поведении ряда.
  \item Декомпозиция (Разложение на компоненты). Фильтрация
    (Разложение на сигнал и шум).
  \item Заполнение пропусков
  \item Обработка выбросов, аномалий, детектирование смены режима
  \item Дезагрегация - попытка определить поведение ряда внутри
    рассматриваемых исходных временных интервалов.
\end{enumerate}

С многомерными рядами:

\begin{enumerate}
  \item То же что и со одним рядом
  \item Классификация
  \item Кластеризация
  \item Поиск взаимосвязей между рядами (расчет корреляций,
    кросс-регрессий, определение причинности)
\end{enumerate}

Вместе с тем также стоит сказать, что в случае временных рядов одни и
те же математические модели могут выполнять несколько функций.
Например, модель скользящего среднего может выступать как предиктором
ряда, так и его фильтром (срезая его высокочастотные колебания,
которые скорее всего являются шумом).

Значительную часть курса мы посвятим техникам прогнозирования
одномерного ряда. О том, что такое (хороший) прогноз временного ряда,
читайте в следующей главе.