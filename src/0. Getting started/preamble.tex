\chapter{Преамбула}
В которой мы кратко опишем, почему данный учебник был создан.

\section{Отсутствие нормальных курсов по временным рядам}

Не существует их. В околофинансовой сфере курсы делятся на три типа - ARIMA/GARCH, Нейронные сети, лайфхаки по работе с временными рядами в машинке.

\subsection{Критика эконометрики}

Задает очень важные вопросы и предлагает очень замшелые ответы.

Эконометрика, в частности временных рядов, полезна в той части, в которой она описывает особенности данных
Например, нам важно знать такие свойства рядов как стационарность, сезонность, наличие эффекта памяти /автокорреляции, кластеризации волатильности, наличие структурных сдвигов, динамика корреляции между активами. Полезны статистические тесты
Зная эти особенности, мы можем грамотно предобработать данные/подобрать модели
В то же время методы предсказания, которые предлагает эконометрика (например, ARIMA/GARCH) практически полностью бесполезны
ЭКОНОМЕТРИКА - ЭТО ДИАГНОСТИКА А НЕ ЛЕЧЕНИЕ!

\subsection{Критика нейронных сетей}

Иногда слепой фит черного ящика сильно вредит. Понимание природы данных усилит надежность/качество моделей.

\subsection{Критика лайфхаков}

Крайне полезно, но не дают полной картины в голове.

\subsection{Кроме того, ...}

Ничего про обработку сигналов. Про фильтр Калмана я чел с эконом образованием услышал случайно и не сразу понял, куда его сунуть в структуру знания.
Много слышал про низкое соотношение сигнал/шум на рынках, но что именно это значит.


\subsection{Зачем случайные процессы}
Реализация случайного процесса — это временной ряд. Без понимания слупов любая попытка понять ряды - бесполезна. Многие делают такую попытку, потому-то большинство курсов и такая лажа.

\subsection{Технический анализ}

Технический анализ - первая попытка человека понять финансовые временные ряды. Некто со смартлаба.
Так может так их и изучать.

\section{Миссия курса}

Поэтому в этом курсе хотелось бы делать упор не на конкретные модели предсказания, а на особенности данных.

"Шопенгауэр никогда не хочет казаться: он пишет для себя, а никто не хочет быть обманутым, и тем более философ, который ставит себе закон: не обманывай никого, и даже самого себя!"

Я пишу этот курс для себя и выглядит он ровно так, как я бы хотел. Он не составлялся для других, но думаю будет полезней всего остального.