\chapter{Поехали}
\url{https://otexts.com/fpp3/intro.html}

\section{Как мы построили данный курс!}
swfgfd
У нас собралось большое количество разнообразных материалов на
темуfadfddf
временных рядов из различных областей науки. Они все
довольно разнообразны, при этом между областями слабое пересечение.
Из-за этого полноценное понимание сутиf
временных рядов не удаётся. Мы поставили задачу
связать все эти области воедино. Мы не претендуем на полноту и
точность знаний: курс во многом авторский. Мы хотим, чтобы к  его
окончанию у студентов развилась интуиция в плане работы с временными
рядами. Мы не ставим задачу, чтобы студенты помнили каждую формулу из
рассмотренных нами, и не планируем рассматривать каждую модель
временного ряда, известную человечеству.

Дело в том, что временные ряды включают элементы следующих систем:
\begin{enumerate}
  \item ТВиМС
  \item Случайные процессы
  \item Технические методы временных рядов
  \item Эконометрические методы временных рядов
  \item Обработка сигналов
  \item Технический анализ
  \item ML
  \item DL
\end{enumerate}

Стратегия: надо накидать структуру курса и затем включать в неё все
имеющиеся материалы. Если что-то не вмещается в структуру, надо
обновить структуру. 

Тактика: выложить на GitLab методичку и лекции. Идти последовательно:
я и ассистенты создаём лекции по одной. Когда заканчиваем одну,
делаем другую. Пишем ноутбуки и сами же разрабатываем домашние
задания (не слишком усложняя их).

Формат курса:
\begin{enumerate}
  \item Методичка. Источником вдохновения является методичка по ПЛА.
    Со ссылками на дополнительные источники: всё равно всё покрыть
    невозможно, главное — дать базу.
  \item Ноутбуки. 16 ноутбуков, соответствующих 15 коммодитис из
    индекса. Цель каждого задания — спрогнозировать ряд с наилучшей
    метрикой качества, используя изученные методы.
  \item Очень хочется обойтись без слайдов.
\end{enumerate}

\section{Что стоит включить}

What can be forecast? Взять 4 пункта и написать, что делать, если они
не выполняются. Для judgmental forecasting дать ссылку на
суперфоркастинг и, возможно, на темы, которые там не покрываются
(если такие есть). Футурология.

Описать, с чем мы будем работать — случаи, когда данные всё-таки есть.

Чек-лист построения (ML) модели. Взять за основу разделы 1.3, 1.4,
1.6. Зачем нужны чек-листы? Не для механического построения моделей,
а чтобы не забывать о важных шагах.

Связь случайных процессов с временными рядами. Глава 1.7, но более
математизированная. Возможно, упомянуть факты вроде того, что
условное матожидание — лучший прогноз для МСК.

Мартингал. Наивный прогноз.
Задание: построить наивный прогноз для простого ряда. Можно
ненавязчиво включать разные официальные источники данных в курс.

TODO: протестировать систему
FIXME: исправить что-то

Кусок для 0 главы:
"
\section{Основы анализа временных рядов}

Среди наиболее распространенных методов анализа временных рядов выделим корреляционный и спектральный анализ, модели авторегрессии и скользящей средней. О некоторых из них речь пойдет ниже.

Если выборка $x_1, x_2, \ldots, x_t, \ldots, x_n$ рассматривается как одна из реализаций случайной величины $X$, временной ряд $x_1, x_2, \ldots, x_t, \ldots, x_n$ рассматривается как одна из реализаций (траекторий) случайного процесса $X(t)$.

Вместе с тем следует иметь в виду принципиальные отличия временного ряда $x_t, (t = 1, 2, \ldots, n)$ от последовательности наблюдений $x_1, x_2, \ldots, x_t, \ldots, x_n$, образующих случайную выборку:

\begin{enumerate}
    \item В отличие от элементов случайной выборки члены временного ряда, как правило, не являются статистически независимыми.
    \item Члены временного ряда не являются одинаково распределенными.
\end{enumerate}

Еще один кусок:
TODO: Переписать своими словами и пофиксить формулы
\textbf{Мы говорили, что стохастический процесс – это функция как бы двух величин: времени и
случайности. Когда случайность фиксирована, мы рассматриваем одну реализацию. Однако все
наши рассуждения касаются характеристик случайных величин, таких как математическое
ожидание M(xt) и другие. Это подразумевает, что в каждый заданный момент времени мы имеем
всю генеральную совокупность случайной величины, соответствующую этому моменту времени.
Всякий раз усреднение идет по повторяющейся выборке. Однако в нашем распоряжении одна
единственная реализация. Поэтому любая статистика, с которой мы будем иметь дело, может
использовать только реализацию, а не повторяющуюся выборку. Единственной, по сути,
возможностью остается усреднение по реализации, по времени. Поэтому нам нужны основания,
чтобы считать, что усреднение по времени в каком-то смысле эквивалентно усреднению по
всевозможным значениям генеральной совокупности.

Процессы, которые обладают таким свойством, называются эргодическими (ergodic).
Недостаточно для процесса быть стационарным, вообще говоря, нужно еще, чтобы процесс был
эргодическим. Иногда это свойство называют свойством хорошего перемешивания.
Разбросанные в разные моменты времени значения временного ряда должны составить такую же
качественно выборку, как повторная выборка в один и тот же момент времени.
Эргодичность – это свойство, позволяющее для оценки математических ожиданий
использовать усреднения по времени (по реализации). Например, мы хотим оценить
математическое ожидание. Мы должны взять всевозможные значения в один и тот же момент
времени t. У нас таких нет. Но у нас есть значения в другие моменты времени. Свойство хорошего
перемешивания означает, что если у нас достаточно длинная реализация, то можно заменить
усреднение по ансамблю, по множеству, усреднением по времени. Для того, чтобы стационарный
процесс был эргодичным, достаточно выполнения следующего условия
(n-1 \sum_{t=1}^{T} (a_t - y_t)^2 
n
K=1
)
𝑛→∞
→ 0.
Мы видим, что стационарные процессы ARMA(p,q) обладают свойством эргодичности.
Нестационарный процесс не может быть эргодическим. Но не всякий стационарный процесс
эргодичен, хотя для практических целей наличие стационарности неявно подразумевает
эргодичность. Важно понимать, что одной стационарности при статистической обработке
реализаций недостаточно.выфв
}
 