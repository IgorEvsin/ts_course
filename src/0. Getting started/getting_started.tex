\chapter{Поехали}
\url{https://otexts.com/fpp3/intro.html}

\section{Как мы построили данный курс}

У нас собралось большое количество разнообразных материалов.
Они все довольно разнообразны, при этом они все являются вещью в себе
- между областями слабое пересечение.
Из-за этого полноценное понимание сути временных рядов не удается.
Мы поставили задачу connecting the dots и связать все эти области воедино.
Мы не претендуем на полноту и точность знаний. Курс во-многом авторский.
Мы хотим, чтобы к его окончанию у студентов развилась интуиция в
плане работы с ними.
Мы не ставим задачу, чтобы студенты помнили каждую форму из рассмотренных нами.
Мы не планируем рассматривать каждую модель временного ряда,
известную человечеству.

Дело в том, что временные ряды включают в себе элементы следующих систем

\begin{enumerate}
  \item ТВиМС
  \item Случайные процессы
  \item Технические методы временных рядов
  \item Эконометрические методы временных рядов
  \item Обработка сигналов
  \item Технический анализ
  \item ML
  \item DL
\end{enumerate}

Стратегия: надо накидать структуру курса и затем в нее запихивать
все материалы, какие у нас есть. Если что-то не вмещается в
структуру, надо обновить структуру.

Тактика: выложить на гитлаб методичку и лекции. Идти с самого начала,
я и ассистенты создаем лекции по одной. Когда заканчиваем одну,
делаем другую. Пишем ноутбуки и сами же делаем домашку (не сильно запариваясь).

Формат курса:

\begin{enumerate}
  \item Методичка. Источником вдохновления является методичка по ПЛА.
    Со ссылками на дополнительные источники чтения, мы все в любом
    случае не успеем покрыть, главное покрыть базу.
  \item Ноутбуки. 16 ноутбуков, соответствующих 15 коммодитис,
    входящих в индекс. Цель каждой домашки - запрогнозить ряд с
    наилучшей метрикой качества, используя изученные методы.
  \item Очень хочется обойтись без слайдов
\end{enumerate}

\section{Что стоит включить}

What can be forecast? Взять 4 пункта и написать, что делать в случае,
если они не выполняются. В случае джаджментал форкастинга дать ссылку
на суперфоркастинг и мб на темы, которые там не покрываются (если такие есть).
Футурология.

Написать, с чем мы будем иметь дело - когда данные все-таки есть.

Чек-лист построения (ML) модели. Взять за основу 1.3, 1.4, 1.6, Зачем
нужны чек-листы (не механизированно строить на их основании модели, а
чтобы не забывать о каких-то важных шагах).

Связь слупов с тайм сериесами. Глава 1.7, только более
математизированная. Может, есть какие-то факты типа как в машинке,
что условное матожидание - лучший прогноз для мск.

Мартингал. Наивный прогноз.
Задание построения наивного прогноза для какого-нибудь простого ряда.
Можно ненавязчиво засовывать разные официальные источники данных и
включать их в курс.