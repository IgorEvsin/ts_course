\chapter{Поехали}
\url{https://otexts.com/fpp3/intro.html}

\section{Как мы построили данный курс}

У нас собралось большое количество разнообразных материалов на
тему
временных рядов из различных областей науки. Они все
довольно разнообразны, при этом между областями слабое пересечение.
Из-за этого полноценное понимание сути
временных рядов не удаётся. Мы поставили задачу
связать все эти области воедино. Мы не претендуем на полноту и
точность знаний: курс во многом авторский. Мы хотим, чтобы к  его
окончанию у студентов развилась интуиция в плане работы с временными
рядами. Мы не ставим задачу, чтобы студенты помнили каждую формулу из
рассмотренных нами, и не планируем рассматривать каждую модель
временного ряда, известную человечеству.

Дело в том, что временные ряды включают элементы следующих систем:
\begin{enumerate}
  \item ТВиМС
  \item Случайные процессы
  \item Технические методы временных рядов
  \item Эконометрические методы временных рядов
  \item Обработка сигналов
  \item Технический анализ
  \item ML
  \item DL
\end{enumerate}

Стратегия: надо накидать структуру курса и затем включать в неё все
имеющиеся материалы. Если что-то не вмещается в структуру, надо
обновить структуру.

Тактика: выложить на GitLab методичку и лекции. Идти последовательно:
я и ассистенты создаём лекции по одной. Когда заканчиваем одну,
делаем другую. Пишем ноутбуки и сами же разрабатываем домашние
задания (не слишком усложняя их).

Формат курса:
\begin{enumerate}
  \item Методичка. Источником вдохновения является методичка по ПЛА.
    Со ссылками на дополнительные источники: всё равно всё покрыть
    невозможно, главное — дать базу.
  \item Ноутбуки. 16 ноутбуков, соответствующих 15 коммодитис из
    индекса. Цель каждого задания — спрогнозировать ряд с наилучшей
    метрикой качества, используя изученные методы.
  \item Очень хочется обойтись без слайдов.
\end{enumerate}

\section{Что стоит включить}

What can be forecast? Взять 4 пункта и написать, что делать, если они
не выполняются. Для judgmental forecasting дать ссылку на
суперфоркастинг и, возможно, на темы, которые там не покрываются
(если такие есть). Футурология.

Описать, с чем мы будем работать — случаи, когда данные всё-таки есть.

Чек-лист построения (ML) модели. Взять за основу разделы 1.3, 1.4,
1.6. Зачем нужны чек-листы? Не для механического построения моделей,
а чтобы не забывать о важных шагах.

Связь случайных процессов с временными рядами. Глава 1.7, но более
математизированная. Возможно, упомянуть факты вроде того, что
условное матожидание — лучший прогноз для МСК.

Мартингал. Наивный прогноз.
Задание: построить наивный прогноз для простого ряда. Можно
ненавязчиво включать разные официальные источники данных в курс.
