\textit{Основано на \cite{adeshereKorrelyaciyaMezhduVremennymi2021},
\cite{panovTeoriyaSluchaynyhProcessov2018}}

\section*{Определение 1.1}

Пусть \( T \) - произвольное множество, \( (\Omega, \mathcal{F},
\mathbb{P}) \) - вероятностное пространство.

Тогда отображение \[ X : T \times \Omega \to \mathbb{R} \] называется
случайной функцией, если для
любого \( t \in T \) функция \[ X(t, \omega) := X_t(\omega) = X_t,
\quad \omega \in \Omega, \]  является случайной величиной на \(
(\Omega, \mathcal{F}, \mathbb{P}) \).

Наиболее важные виды случайных функций:
\begin{enumerate}
  \item случайные процессы (stochastic processes, random processes):
    \( T \subset \mathbb{R} \)
    \begin{enumerate}
      \item случайные процессы с дискретным временем: \( T =
        \mathbb{Z}_+ \) (иногда \( T = \mathbb{Z} \))
      \item случайные процессы с непрерывным временем: \( T = [0,
        \infty) \) (иногда \( T = \mathbb{R} \))
    \end{enumerate}
  \item Cлучайные поля: \( T \subset R^k, \quad k \geq 2 \).
\end{enumerate}

\section*{Определение 1.2}
Траекторией случайного процесса \( X_t \)
называется отображение \( t \mapsto X_t(\omega) \) при
фиксированном \( \omega \).
Конечномерным распределением случайного процесса \( X_t \)
называется распределение вектора \( (X_{t_1}, X_{t_2}, \ldots,
X_{t_n}) \) для фиксированного набора моментов
времени \( t_1, t_2, \ldots, t_n \).

\section*{Определение 1.3} Временным рядом \( Y_t \) называется
набор значений траектории случайного процесса,
измеренных в наблюдаемые моменты времени. Он представляет собой
дискретизированную или наблюдаемую версию траектории случайного процесса.

Эквидистантный временной ряд — это последовательность измерений,
сделанных через равные промежутки времени. Однако на практике
интервалы между измерениями часто бывают неравными, особенно в
экономике или при наблюдении за природными явлениями. Например,
данные могут собираться только в рабочие дни, что приводит к
пропускам на выходные и праздники. Эти пропуски нельзя
игнорировать, особенно если временные лаги играют важную роль
(например, при доставке скоропортящихся товаров).
Такие увеличенные интервалы не всегда можно просто выбросить, особенно
если речь идет о непрерывных производствах или фиксированных временных лагах.
Например, если для производства товара А нужен скоропортящийся
товар Б, на доставку которого надо 3 дня, то рост производства Б в
понедельник и вторник отобразится в товаре А на третий рабочий
день, а рост производства Б в пятницу - уже в понедельник (а это
следующая точка, если выходные отброшены).

Главное отличие временного ряда от других типов данных — важность
временного порядка. Не только само значение, но и момент его
измерения имеют значение.

Проводя наблюдения за
каким-то природным явлением, мы вовсе не извлекаем получаемые
значения из одной и той же генеральной совокупности. Даже если
настройки прибора и положение датчиков не менялись, состояние
измеряемого объекта в каждый новый момент времени будет другое.
Попросту говоря, это будет уже другая случайная величина. Серию
измерений, выполненных одним и тем же прибором, даже неподвижно
стоящим под одной и той же горой, нельзя рассматривать, как серию
выборок из одного и того же пространства элементарных событий. Для
описания случайного процесса, в отличие от случайной величины,
недостаточно задать его функцию
распределения один раз. Просто потому, что в разные моменты времени
t она может быть разной. А еще для случайного процесса надо
определить функцию совместного распределения вероятностей для
моментов времени t и t+dt и так далее.

Чтобы оценить эти функции,
наблюдая за случайным процессом, нужна не одна реализация, а целый
ансамбль. Ну, хотя бы десяток реализаций. Причем, это обязательно
должны быть реализации одного и того же случайного процесса. Тогда
и только тогда для каждого момента времени у нас будет несколько
измерений одной и той же случайной величины. Как их обрабатывать
дальше, мы уже знаем из школьного вузовского курса статистики.

Это часто игнорируется при применении
стандартных статистических методов, что может привести к ошибочным
выводам, например, к ложной корреляции между числом пиратов и
глобальным потеплением.

Но что же делать, если у нас есть только одна Земля? Как изучать
взаимосвязи между процессами, каждый из которых мы наблюдаем в
единственном экземпляре?! Проблема эта невероятно сложная,
практически нерешаемая. Мы еще вернемся к ней, когда будем говорить
о свойствах случайных процессов, в частности о стационарности и эргодичности.
Оказывается, что для некоторых классов случайных
процессов, все характеристики которых неизменны во времени, наличие
ансамбля не обязательно! То есть, нам не потребуется десять
реализаций, чтобы оценить какую-нибудь статистику. Вместо этого
достаточно некоторое время понаблюдать за одной! Например, чтобы
оценить коэффициент корреляции между X и Y, достаточно иметь одну
реализацию X и еще одну – Y. Что, собственно, все мы и делаем,
когда вычисляем коэффициент корреляции между потеплением и
пиратами.

Но что же делать с остальными рядами, теми которые не имеют данных
приятных свойств. Рядами с трендами, сезонными и суточными
циклами, и т.д.? Как искать связь между ними и оценивать ее
значимость? На этот вопрос мы в том числе попытаемся найти ответ в
данном учебнике.

Пока же следует запомнить, что общая проблема алгоритмов обработки
временных рядов — это отсутствие математической строгости. Ведь даже когда
мы используем для анализа экспериментальных сигналов алгоритмы со
строгим обоснованием и доказанной оптимальностью, у нас всегда
остается открытым вопрос о том, не нарушены ли условия применимости
таких алгоритмов? Ведь любой алгоритм всегда начинается с преамбулы
(требования), что исходные данные должны обладать вполне
определенными свойствами. Но когда мы имеем дело с
экспериментальным сигналом, доказать выполнение этих требований
почти невозможно. Поэтому было бы наивно думать, что корректность
результата можно гарантировать строгостью метода. На практике
использование строгих методов не дает никаких преимуществ, если с
тем же уровнем строгости не доказана адекватность модели данных, в
рамках которой сформулирован метод. При режимных наблюдениях это
почти невозможно. В лучшем случае можно только предполагать, что
«базовая модель» сигнала вполне адекватна реальным данным. В худшем
(и, к сожалению, более типичном) случае, наоборот, имеются видимые
несоответствия между требованиями теоретической модели и
экспериментальным сигналом. Но если у нас нет уверенности в
адекватности используемой модели данных, это ставит под сомнение и
все результаты, полученные в рамках такой модели.

С одной стороны, такая постановка проблемы может обескуражить -
кажется, что не существует универсального рецепта построения
идеальной модели временного ряда.

С другой стороны, все это дает право на гибкость и вольность мысли.
При изложении материала мы бы хотели, чтобы читатели постоянно
держали в голове связь между временным рядом и свойствами
породившего его случайного процесса.
Мы верим в то, что именно понимание специфических
свойств временного ряда способно привести к пониманию, как
построить наиболее точный и устойчивый прогноз. Как следствие, мы
не хотим делать слишком большой упор на разъяснение механизмов
работы конкретных моделей временных рядов или алгоритмов машинного
обучения. Вместо этого мы бы хотели, чтобы у читателей
сформировалось понимание, какие особенности исходных данных
пытались решить создатели того или иного алгоритма. Сфокусироваться
на исходном problem space, а не на текущем solution space
(тем более, что он постоянно меняется, появляются все новые алгоритмы
  предобработки данных, все новые алгоритмы машинного обучения, схемы
обучения и т.д.).

\printbibliography[heading=subbibliography, title={Источники}]