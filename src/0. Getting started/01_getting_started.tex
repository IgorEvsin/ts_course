\chapter{Поехали}
\url{https://otexts.com/fpp3/intro.html}

\section{Как мы построили данный курс}

Привет! Мы команда увлеченных анализом данных энтузиастов, которых в
какой-то момент объединило благородное желание освоить анализ
временных рядов. У нас собралось большое количество
разнообразных материалов на
тему временных рядов из различных областей науки.

Временные ряды - это непросто. Во-многом, дело в том, что основы
анализа временных рядов затрагиваются
в рамках любой из приведенных ниже дисциплин:
\begin{enumerate}
  \item Теория вероятностей и математическая статистика
    \begin{enumerate}
      \item Во временных рядах есть своя специфика, из-за которой
        слепое применение классических статистических методов может
        привести к ошибочным выводам
    \end{enumerate}
  \item Случайные процессы
    \begin{enumerate}
      \item Очень важно, однако сложность изучения не окупается
        реальными результами
    \end{enumerate}
  \item Технические методы временных рядов
    \begin{enumerate}
      \item Ранние и во-многом неформализованные способы обработки
        последовательностей данных
    \end{enumerate}
  \item Эконометрические методы временных рядов
    \begin{enumerate}
      \item Эконометрика, в частности временных рядов, полезна в той части, в
        которой она описывает особенности данных
      \item Например, нам важно знать такие свойства рядов как стационарность,
        сезонность, наличие эффекта памяти /автокорреляции, кластеризации
        волатильности, наличие структурных сдвигов, динамика корреляции между
        активами. Полезны статистические тесты
      \item Зная эти особенности, мы можем грамотно предобработать
        данные/подобрать модели
        В то же время методы предсказания, которые предлагает эконометрика
        (например, ARIMA/GARCH) практически полностью бесполезны
      \item Следовательно эконометрику следует рассматривать скорее
        как полезное средство для диагностики, а не решение.
    \end{enumerate}
  \item Теория обработки сигналов
    \begin{enumerate}
      \item Достаточно зрелая и полезная дисциплина, о которой
        некоторые из нас, в силу своего экономического образования,
        ничего не знали, за исключением каких-то отрывков знаний
        вроде фильтра Калмана.
    \end{enumerate}
  \item Технический анализ
    \begin{enumerate}
      \item "Технический анализ - первая попытка человека понять финансовые
        временные ряды" - некто со смартлаба.
    \end{enumerate}
  \item Курсы по машинному обучению / глубокому обучению
    \begin{enumerate}
      \item Иногда слепой фит черного ящика сильно вредит. Понимание природы
        данных усилит надежность/качество моделей.
    \end{enumerate}
\end{enumerate}

И это мы еще не говорим уже о сотнях статей в Интернете, посвященных полезным
лайфхакам для работы с рядами. Зачастую в них можно почерпнуть
действительно полезную информацию, но полной картины они, разумеется, не дают.

Как можно заметить, дисциплин, косвенно касающихся идеи временных
рядов довольно много, они разнообразны, при этом между некоторыми областями
наблюдается довольно слабое пересечение.
Из-за этого полноценное понимание сути временных рядов оказывается
затруднено. Поэтому когда мы составляли
данный курс, основной антицелью являлось не допустить тупого
пересказа имеющейся литературы в этих областях. Вместо этого мы
хотели бы, чтобы к  его окончанию у студентов развилась интуиция по
работе с временными рядами. Мы не претендуем на полноту и
точность знаний: курс во многом авторский. Мы также не ставим задачу,
чтобы студенты помнили каждую формулу из
рассмотренных нами, и не планируем рассматривать каждую модель
временного ряда, известную человечеству.

Мы религиозно верим в формулу "знать -- значит уметь", поэтому горячо
рекомендуем прорешать подготовленные нами
\href{https://github.com/IgorEvsin/ts_examples}{практические
упражнения по временным рядам.} Мы очень старались, когда создавали
его, так что можете похвалить нас звездочкой на гитхабе.
