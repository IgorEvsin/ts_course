\chapter{Технические особенности временных рядов}

\defn{Эквидистантность}{
  Эквидистантный временной ряд -- это последовательность измерений,
  сделанных через равные промежутки времени.
}

\ex{}{
  На практике интервалы между измерениями часто бывают неравными, особенно в
  экономике или при наблюдении за природными явлениями. Например,
  данные могут собираться только в рабочие дни, что приводит к
  пропускам на выходные и праздники. Эти пропуски нельзя
  игнорировать, особенно если временные лаги играют важную роль
  (например, при доставке скоропортящихся товаров).
  Такие увеличенные интервалы не всегда можно просто выбросить, особенно
  если речь идет о непрерывных производствах или фиксированных временных лагах.
  Например, если для производства товара А нужен скоропортящийся
  товар Б, на доставку которого надо 3 дня, то рост производства Б в
  понедельник и вторник отобразится в товаре А на третий рабочий
  день, а рост производства Б в пятницу - уже в понедельник (а это
  следующая точка, если выходные отброшены).
}