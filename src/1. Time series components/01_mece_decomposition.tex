\chapter{Компоненты временных рядов}
\label{chap: time_series_components}

\section{Мотивация}

Зачем это нужно, про консалтеров немного и ссылки

\section{MECE декомпозиция}

\label{sec: mece_decomposition}

\ex{}{
  Если перед вами стоит задача предсказания прибыли компании,
  имеет смысл предварительно
  представить ее в виде разности выручки и затрат и далее
  отдельно прогнозировать
  каждую компоненту. Вспомним базовую микроэкономику:

  \[
    \begin{aligned}
      \text{Profit} &= \text{Total Revenue} - \text{Total Costs} \\
      \text{Total Revenue} &= \text{Average Check} \times
      \text{Quantity Sold} \\
      \text{Total Costs} &= \text{Variable Costs} + \text{Fixed Costs} \\
      \text{Variable Costs} &= \text{Average Variable Cost}
      \times \text{Quantity Sold} \\
      \text{Profit} &= (\text{Average Check} - \text{Average
      Variable Cost}) \times
      \text{Quantity Sold} - \text{Fixed Costs}
    \end{aligned}
  \]

  Разложив прибыль в итоговом виде, мы можем создать
  предсказательную модель для каждого элемента и затем получить
  итоговый прогноз прибыль, собрав в композицию прогнозы.
}