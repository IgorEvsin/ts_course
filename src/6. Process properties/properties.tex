\chapter{Свойства случайных процессов}
\section*{Часть III} \textbf{Свойства случайных процессов}

\textit{Основано на \cite{adeshereKorrelyaciyaMezhduVremennymi2021},\cite{panovTeoriyaSluchaynyhProcessov2018}}

\section{Стационарность} Afdfg
\subsection*{Определение 9.1} Случайный процесс \( X_t \) называется
стационарным (stationary, стационарным в узком смысле), если все его
конечномерные распределения инвариантны относительно сдвигов, т.е.
для любых наборов моментов времени $t1 , \dots, tn$ , любых вещественных
$x1 , \dots, xn$ и любого $h > 0$:

\[ \mathbb{P}(X_{1} \leq x_{1}, \ldots, X_{n} \leq x_{n}) =
\mathbb{P}(X_{t_1 + h} \leq x_{1}, \ldots, X_{t_n + h} \leq x_{n}), \]

\subsection*{Определение 9.2} Случайный процесс \( X_t \) называется
стационарным стационарным в широком смысле (wide sense stationary,
  weakly stationary, covariance stationary,
second-order stationary), если $m(t)$ является постоянной величиной
(не зависящей от $t$), и кроме того, для любых $h > 0$, $s$, $t \in R$ выполнено

\[ K(t + h, s + h) = K(t, s). \]

Другими словами, процесс с постоянным математическим ожиданием
является стационарным в широком смысле тогда и только тогда, когда
существует функция $ \gamma : \mathbb{R} \to \mathbb{R} $ такая, что $
K(t, s) = \gamma(t-s). $ для любых $t$, $s$. Функция $ \gamma
(\cdot) $ называется автоковариационной функцией
(autocovariance function)
и обладает следующими свойствами.

\textbf{Утверждение 9.3.} Пусть $\gamma$ — автоковариационная функция
некоторого стационарного в широком смысле случайного процесса. Тогда
\begin{itemize}
  \item[(i)] $\gamma(0) \geq 0$.
  \item[(ii)] $|\gamma(u)| \leq \gamma(0)$.
  \item[(iii)] $\gamma$ является четной функцией.
\end{itemize}

\section{Эргодичность}
Понятие эргодичности мотивировано законом больших чисел. Рассмотрим процесс
$X_t$, наблюдаемый в дискретные моменты времени $t = 1, 2, \ldots, T$
и зададимся вопросом, сходится ли процесс \[ M_T = \frac{1}{T}
\sum_{t=1}^{T} X_t \] при устремлении горизонта времени $T \to
\infty$.

\textbf{Определение 11.1.} Процесс $X_t$ с дискретными
временами $t = 1, 2, \ldots$ называется эргодическим, если \[ M_T
\xrightarrow{p} \mu, \quad \text{при} T \to \infty \] где $\mu$ —
некоторая константа.

Давайте вернемся к тому, с чего начинали. Во введении мы говорили о
необходимости иметь множество реализаций случайного процесса для
того, чтобы иметь возможность адекватно посчитать по ним статистики.
Оказывается, что для эргодических случайных
процессов, все характеристики которых неизменны во времени, наличие
ансамбля не обязательно! То есть, нам не потребуется десять
реализаций, чтобы оценить какую-нибудь статистику. Вместо этого
достаточно некоторое время понаблюдать за одной! Например, чтобы
оценить коэффициент корреляции между X и Y, достаточно иметь одну
реализацию X и еще одну – Y. Что, собственно, все мы и делаем,
когда вычисляем коэффициент корреляции между потеплением и
пиратами.

Иногда различают эргодичность по среднему, по дисперсии и т.д. При
анализе наблюдений очень часто априори считается, что исследуемые
процессы являются эргодическими. Иногда это даже не оговаривается специально.

Но если мы хотим избежать грубейших ошибок, то нельзя забывать, что
гипотеза эргодичности – это только гипотеза. Подавляющее
большинство долговременных наблюдений продолжается конечное время
(вы поняли, это такая шутка), а на выходе получается единственный
ряд. Доказать эргодичность такого процесса в принципе невозможно.
Поэтому, начиная анализ данных, мы чаще всего просто постулируем ее
явным образом или неявно. А что еще остается делать, если в наличии
куча данных и руки чешутся начальник требует срочно использовать
всю мощь безупречного, многократно проверенного теоретиками
статистического инструментария для достижения практических целей?

На самом деле, подавляющее большинство временных рядов вовсе
не являются эргодическими. И если доказать эргодичность процесса
достаточно сложно (практически нереально), то вот
опровергнуть ее часто можно без особых усилий. Достаточно просто
вспомнить, что практически все экспериментальные временные ряды
существенно нестационарны. Огромный массив накопленных
экспериментальных данных однозначно свидетельствует, что априорная
"базовая модель" почти любого природного процесса – это вовсе не
белый шум (для которого действительно можно заигрывать с
эргодичностью). Нет, спектры большинства реальных сигналов имеют
степенной вид (Точнее, они обычно становятся степенными после
удаления доминирующих периодичностей - сезонной, суточной и т.д.)

Но если ряд не стационарен, то он заведомо не может
рассматриваться, как последовательность измерений одной и той же
случайной величины. Для него совершенно бессмысленно оценивать те
статистики, которые вводятся и исследуются при анализе случайных величин.

Из стационарности не следует эргодичность! Привести пример.
